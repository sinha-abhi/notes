% Notes on implementing a subtype of the AbstractArray interface in Julia.
%   Mainly for keeping track of what I need to do in my implementation of 
%   a Julia package for Elias-Fano Indexing (otherwise known as Quasi-Succinct
%   Indexing).
%
%   For official documentation, see
%   <https://docs.julialang.org/en/v1/manual/interfaces/#man-interface-array-1>.
%
% Abhi Sinha, sinha45@purdue.edu
% 2019-12-17
\documentclass[10pt]{article}

%--
% Ubuntu preamble location 
%\input{/home/abhi/abhi/personal/git-repos/config/preamble.tex}

% Fedora preamble location 
\input{/home/abhi/abhi/git-repos/configuration/preamble.tex}

% Darwin preamble location 
%\input{/Users/abhisinha/git/configuration/preamble.tex}
%--

\title{Implementation Notes}
\subject{Elias-Fano in Julia}
\date{December 17, 2019}

\hypersetup{unicode = true,
            pdftitle = {abstract-array-julia},
            pdfborder = {0 0 0},
            breaklinks = true}

\begin{document}
\maketitle

\section{Elias-Fano Implementation Notes in Julia}\label{ef-impl-julia}
\subsection{Source}\label{source}
The contents of this document were compiled from
\begin{itemize}
\item
    Julia's official docs (\url{https://docs.julialang.org/en/v1/}), and 
\item
    Julia's source code on GitHub (\url{https://github.com/JuliaLang/julia}).

\end{itemize}
\subsection{\texttt{AbstractArray} Interface in Julia}\label{abstract-array}
\texttt{EFArray} should be of subtype \texttt{AbstractArray}.
\begin{itemize}
\item
    Required methods:
    \begin{itemize}
    \item
        \texttt{size(A)}
    \item
        \texttt{getindex(A, i::Int)}
    \item
        \texttt{setindex!(A, v, i::Int)}
    \end{itemize}
\item
    Optional methods:
    \begin{itemize}
    \item
        \texttt{IndexStyle(::Type)}
    \item
        \texttt{getindex(A, I...)}
    \item
        \texttt{setindex!(A, I...)}
    \item
        \texttt{iterate}
    \item
        \texttt{length(A)}
    \item
        \texttt{similar(A)}
    \item
        \texttt{similar(A, ::Type\{S\})}
    \item
        \texttt{similar(A, dims::Dims)}
    \item
        \texttt{similar(A, ::Type\{S\}, dims::Dims)}
    \end{itemize}
\item
    \todo what is the expected behavior of these functions?
\end{itemize}

\subsection{\texttt{IndexStyle}}\label{index-style}
Array data structures are usually defined in one of two ways:
\begin{enumerate}
\def\labelenumi{\arabic{enumi}.}
\item
    Use one index to efficiently access an array's elements -- known as linear
    indexing
\item
    Use indices specified for every dimension to intrinsically access
    an array's elements
\end{enumerate}
In Julia, \texttt{IndexLinear()} is used to declare arrays of the first type,
and \texttt{IndexCartesian()} is used for arrays of the second type.

\subsubsection{\texttt{IndexLinear()}}\label{index-linear}
\texttt{IndexLinear()} requires only \texttt{getindex(A::ArrayType, i::Int)}.
\begin{itemize}
\item
    If the array is indexed with multidimensional indices, the fallback \\
    \texttt{getindex(A::ArrayType, I...)()} converts the indices
    into one linear index and calls the previous method.
\end{itemize}

\subsubsection{\texttt{IndexCartesian()}}\label{index-cartesian}
\texttt{IndexCartesian()} requires methods to be defined for each supported
dimensionality with \texttt{ndims(A) Int} indices.

\subsection{Parameters of \texttt{AbstractArray}}\label{params}
There are two important parameters when defining a subtype of
\texttt{AbstractArray}: \texttt{eltype} and \texttt{ndims}.
Defining these two parameters and the three required methods 
(see \cref{abstract-array})
allows our subtype to act as a fully functioning \texttt{Array}.
(The most important qualities being indexable and iterable.)
\begin{enumerate}
\def\labelenumi{\arabic{enumi}.}
\item
    \texttt{eltype(type)}
    \begin{itemize}
    \item
        Type of elements generated by iterating over a collection
        of the given \texttt{type}
    \item
        New types must define \texttt{eltype(::Type)}.
    \item
        The declaration
        \begin{center}
            \texttt{eltype(x) = eltype(typeof(x))}
        \end{center}
        is provided to allow an instance to be passed instead of a type.
    \end{itemize}
\item
    \texttt{ndims(A::AbstractArray) -> Integer}
    \begin{itemize}
    \item
        Returns the number of dimensions of \texttt{A}
    \end{itemize}
\end{enumerate}

\subsection{Other Specific Implementation Notes}\label{specific-notes}
\todo

\end{document}

