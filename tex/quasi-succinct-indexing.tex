% Notes on Quasi-Succinct Indexing
%   Adapted from `Quasi-Succinct Indices` by Sebastiano Vigna at 
%   https://arxiv.org/abs/1206.4300
% 
% Abhinav Sinha, sinha45@purdue.edu
% 2019-12-12
\documentclass[10pt]{article}

%--
% Ubuntu preamble location 
%\input{/home/abhi/abhi/personal/git-repos/config/preamble.tex}

% Fedora preamble location 
\input{/home/abhi/abhi/git-repos/configuration/preamble.tex}

% Darwin preamble location 
%\input{/Users/abhisinha/git/configuration/preamble.tex}
%--

\title{Notes}
\subject{Quasi-Succinct Indexing}
\date{December 12, 2019}

\hypersetup{unicode = true,
            pdftitle = {quasi-succinct-indexing},
            pdfborder = {0 0 0},
            breaklinks = true}

\begin{document}
\maketitle

\section{Quasi-Succinct Indexing}\label{quasi-succinct-indexing}
\subsection{Index Compression}\label{index-compression}
\begin{itemize}
\item
    Instantaneous codes -- storage of integers is proportional to size of integer
    \begin{itemize}    
    \item
        smaller numbers use fewer bits
    \end{itemize}
\item
    Gap encoding -- turns lists of increasing numbers into lists of smaller integers
    \begin{itemize}
    \item
        smaller integers being the gaps between successive values
    \end{itemize}
\end{itemize}

\subsection{Brief Overview of Unary-Code}\label{unary-code}
Unary code represents a natural number \(n\) as
\begin{itemize}
\item
    \(n\) 1's followed by a 0 -- for \(n\) \textit{non-negative}, or
\item
    \(n - 1\) 1's followed by a 0 -- for \(n\) \textit{strictly positive}.
\end{itemize}
We can exchanges 0's and 1's without loss of generality.
The flipped version is often called negated unary code.
\underline{Example}: 
\( 5 \rightarrow \texttt{111110} \text{ or } \texttt{000001} \)

\subsection{Representation of a Monotone Sequence}\label{monotone-sequence}
Suppose we have the following monotone sequence for \(x_i \in \NN\):    
\[ x_0 \leq x_1 \leq \ldots \leq x_{n-1} \leq u, \]
where \(u\) is some upper bound. 
A question to consider: what is the most effective way to choose \(u\)?
Two possible options are to choose \(u\) s.t.
    (1) \(\log{u} \in \NN\), or
    (2) \(u = x_{n-1} + 1\).

\begin{itemize}
\item
    We store the sequence in a ``high/low bit representation'' using two \textit{bit-arrays}.
    \begin{itemize}
    \item
        The lower \(l = \max{\lbrace 0, \left \lfloor\log{\frac{u}{n}} \right \rfloor \rbrace }\) 
        bits of each \(x_i\) are stored \textit{explicitly} and \textit{contiguously} in a 
        \textbf{lower-bits array}.
    \item
        The remaining upper-bits are stored as a sequence of \textit{unary code gaps} in an
        \textbf{upper-bits array}.
        \begin{itemize}
        \item
            difference in upper bits 
            \(= \left \lfloor \frac{x_i}{2^l} \right \rfloor 
              - \left \lfloor \frac{x_{i-1}}{2^l} \right \rfloor\).
        \item
            For the sake of completeness, let \(x_{-1} = 0\).
        \end{itemize}
    \end{itemize}
\item
    This representation uses at most \(2 + \left \lceil \log{\frac{u}{n}} \right \rceil\) bits
    \textbf{per element} meaning that this representation is \textbf{quasi-succinct}.
    \begin{proof}
    Each unary encoding uses 1 stop bit, and each other written bit increases the value 
    of the upper bits by \(2^l\). 
    This cannot happen more than
    \( \left \lfloor \frac{x_{n-1}}{2^l} \right \rfloor \) times.
    This leads us to the following:
    \[ \left \lfloor \frac{x_{n-1}}{2^l} \right \rfloor 
       \leq \left \lfloor \frac{u}{2^l} \right \rfloor
       \leq \frac{u}{2^l} \leq 2n. \]
    Unless \(\frac{u}{n}\) is a power of 2, we have 
    \[ \left \lceil \log{\frac{u}{n}} \right \rceil 
       = \left \lfloor \log{\frac{u}{n}} \right \rfloor + 1. \]
    If \(\frac{u}{n}\) is a power of 2, then 
    \[ \left \lceil \log{\frac{u}{n}} \right \rceil = \log{\frac{u}{n}}, \]
    but the previous equation is bounded by \(n\) instead of \(2n\).
    The informational-theorectial limit is
    \[ \left \lceil \log{\binom{u + n}{n}} \right \rceil 
       \approx n \log{\left( \frac{u + n}{n} \right) }, \]
    so we conclude that this representation is close to succinct.
    \end{proof}
\end{itemize}

\subsection{Example}\label{example}
Suppose we have a list \texttt{[5, 8, 8, 15, 32]} with \(u= 36\).
We compute \(l = \left \lfloor \log{\frac{36}{5}} \right \rfloor = 2 \).
Split the lower \(l\) bits of the binary representation of each element of the list.
\begin{verbatim}
    5           8           8          15            32
+---+----+ +----+----+ +----+----+ +----+----+ +------+----+
| 1 | 01 | | 10 | 00 | | 10 | 00 | | 11 | 11 | | 1000 | 00 |
+---+----+ +----+----+ +----+----+ +----+----+ +------+----+
\end{verbatim}
The lower-bits array:
\begin{verbatim}
+----+----+----+----+----+
| 01 | 00 | 00 | 11 | 00 |
+----+----+----+----+----+
\end{verbatim}
To compute the upper-bits array, we find the difference between each of the upper bits
and then convert to unary code.
\begin{verbatim}
    5           8           8          15            32
+---+----+ +----+----+ +----+----+ +----+----+ +------+----+
| 1 | 01 | | 10 | 00 | | 10 | 00 | | 11 | 11 | | 1000 | 00 |
+---+----+ +----+----+ +----+----+ +----+----+ +------+----+
  |          |           |           |            |
  |          |           |           |            |
  v          v           v           v            v
  1          2           2           3            8
\end{verbatim}
Recall that we choose \(x_{-1} = 0\) for consistency.
The upper-bits array:
\begin{verbatim}
  1     1   0   1      5
+----+----+---+----+--------+
| 01 | 01 | 1 | 01 | 000001 |
+----+----+---+----+--------+
\end{verbatim}

\subsection{Recovering \texorpdfstring{\(x_i\)}{xi}}\label{recovering-xi}
\begin{itemize}
\item
    Perform \(i\) unary code reads in the upper-bits array to \(p^{th}\) bit position
    \begin{itemize}
    \item
        The value of the upper bits is exactly \(p - i\)
    \item
        \underline{Example}: To read the upper bits of 15, we perform 4 unary code reads to 
        position 7 of the upper-bits array.
        Thus, the value of the upper bits of 15 is \(7 - 4 = 3\).
    \end{itemize}
\item
    Extract lower bits with random access at position \(il\) in the lower-bits array
\end{itemize}

\subsection{Some Optimizations for Accessing}\label{optimizations}

\subsubsection{Forward Pointers}\label{forward-pointers}
\begin{itemize}
\item
    Choose a \textit{quantum} \(q\).
\item
    Store \textbf{forward pointers} at positions that we would reach after 
    \(kq\) unary code reads.
    \begin{itemize}
    \item
        This is the position immediately after \(kq - 1\) bits.
    \end{itemize}
\item
    Retrieve \(x_i\) by simulatinging \(q \left \lfloor \frac{i}{q} \right \rfloor\) reads
    then \(i \mod q\) sequential reads.
\item
    On average, recovering \(x_i\) takes constant time.
\item
    Memory wise, recovery takes at most \(3q\) bits.
\item
    A smaller \(q\) takes less reads, but requires more space.
\end{itemize}
\subsubsection{Skip Pointers}\label{skip-pointers}
\begin{itemize}
\item
    Store positions reached at negated unary code bits.
    \begin{itemize}
    \item
        In other words, store pointers to the positions of each \texttt{1} in the
        upper-bits array.
    \end{itemize}
\end{itemize}
A similar analysis of time and space complexity can be done, but I don't really feel
like doing that right now. This method may not even be that helpful for what I want.

\subsection{Quasi-Succinct Bitstream}\label{quasi-succinct-bitstream}
\subsubsection{Bitstream Layout}\label{bitstream-layout}
The desired Bitstream Layout is
\begin{verbatim}
+----------+-----+-----+---------+-----+-----+---------+-------+
| metadata | p_0 | ... | p_{s-1} | l_0 | ... | l_{n-1} | u ... |
+----------+-----+-----+---------+-----+-----+---------+-------+
                        Bitstream Layout
\end{verbatim}
\begin{itemize}
\item
    \(n\): number of elements
\item
    \(u\): upper bound of \(x_i\)
\item
    \(l = \max{\lbrace 0, \left \lfloor\log{\frac{u}{n}} \right \rfloor \rbrace } \):
    number of bits of each element stored in lower-bits array
\item
    \(q\): arbitrary quantum
\item
    \(m = n + \left \lfloor \frac{u}{2^l} \right \rfloor \):
    maximum length of the \textit{upper-bits array}
\item
    \(w = \left \lceil \log{(m + 1)} \right \rceil  \):
    width of the pointers
    \begin{itemize}
    \item
        If \(u\) is not known, then the metadata must contain data
        to compute \(w\).
    \end{itemize}
\item
    \(s = \left \lfloor \frac{n}{q} \right \rfloor \):
    number of forward pointers
    \begin{itemize}
    \item
        \(s \leq \left \lfloor \frac{m}{q} \right \rfloor \)
        for skip pointers that point to negated unary codes
    \end{itemize}
\item
    \(d + sw\): 
    beginning position of lower-bits array
\item
    \(sw + nl + d\): 
    beginning position of upper-bits array
\end{itemize}

\subsubsection{Information in the Metadata}\label{metadata}
We want to store \(n\), \(l\), \(q\), and \(w\) in the metadata in some order.
Order does not matter, as long as, it is consistent and clearly marked in the implementation.

\end{document}
